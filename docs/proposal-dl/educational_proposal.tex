\documentclass[12pt, a4paper]{article}
\usepackage{mathptmx} % Times New Roman font
\usepackage[utf8]{inputenc}
\usepackage[T1]{fontenc}
\usepackage{geometry}
\usepackage{titlesec}
\usepackage{hyperref}
\usepackage{graphicx}
\usepackage{float}
\usepackage{tocloft}

\setlength{\cftbeforesecskip}{1.5em}
\setlength{\cftsecnumwidth}{2em}

\geometry{margin=1in}

\begin{document}

\begin{titlepage}
    \centering
    \vspace*{1cm}
    
    \includegraphics[width=0.6\textwidth]{binuslogo.png}
    
    \vspace{1.5cm}
    
    {\Large \textbf{BINUS University} \par}
    \vspace{0.5cm}
    {\Large School of Computer Science \par}
    
    \vspace{2cm}
    
    {\LARGE \textbf{Proposal: Gamified Doodle Recognition for English Vocabulary Acquisition in Indonesia} \par}
    
    \vspace{2cm}
    
    \begin{tabular}{ll}
    Valent Nathanael & 2702343706 \\
    Joceline Araki & 2702338334 \\
    Lauw, Samuel Lelono & 2702348026 \\
    \end{tabular}
    
    \vfill
\end{titlepage}

\renewcommand{\cfttoctitlefont}{\hfill\Large\bfseries}
\renewcommand{\cftaftertoctitle}{\hfill}
\tableofcontents
\newpage

\section{Executive Summary}
This proposal outlines a strategic enhancement of the existing \texttt{doodle\_app.py} application to serve a dual purpose: facilitating English vocabulary acquisition for Indonesian children and stress-testing the robustness of the ResNet classification model. By leveraging the "Challenge Mode" and the immediate visual feedback loop, the application can transform abstract language learning into a concrete, engaging visual-motor activity.

\section{Educational Context: English for Indonesian Learners}
English proficiency is a critical skill in Indonesia's educational landscape. However, traditional rote memorization of vocabulary can be disengaging for young learners.

\subsection{Visual-Motor Learning Strategy}
The "Doodle Recognition" app provides a unique pedagogical approach:
\begin{itemize}
    \item \textbf{Active Recall:} Instead of passively reading a word, the child must visualize the object (e.g., "CAT") and actively draw it.
    \item \textbf{Immediate Feedback:} The model's real-time prediction provides instant validation. If the app recognizes the doodle as a "CAT", the child receives positive reinforcement, strengthening the association between the English word and the concept.
    \item \textbf{Gamification:} The existing scoring and streak systems in \texttt{doodle\_app.py} introduce a competitive element that encourages repeated practice and sustained engagement.
\end{itemize}

\subsection{Localized Adaptation}
To specifically target Indonesian learners, the proposal suggests:
\begin{itemize}
    \item \textbf{Bilingual Prompts:} Displaying the target word in English (e.g., "APPLE") with a smaller Indonesian subtitle (e.g., "APEL") to bridge the language gap.
    \item \textbf{Culturally Relevant Categories:} Prioritizing categories that are familiar in the Indonesian context (e.g., common tropical fruits, local animals) during the "Challenge Mode".
\end{itemize}

\section{Technical Objective: Model Robustness Testing}
Using the application with children provides a rigorous test environment for the underlying ResNet machine learning model.

\subsection{The Challenge of "Child-Drawings"}
Children's drawings often differ significantly from the training data (typically collected from adults or general public in datasets like QuickDraw):
\begin{itemize}
    \item \textbf{Abstraction Levels:} Children may focus on different salient features than adults.
    \item \textbf{Motor Control:} Jittery lines and unclosed shapes are more common.
    \item \textbf{Scale and Perspective:} Non-canonical viewpoints (e.g., "flat" representations).
\end{itemize}

\subsection{Evaluating ResNet Architecture}
The application utilizes a ResNet (Residual Network) architecture for image classification. ResNet is a deep convolutional neural network designed to solve the problem of vanishing gradients in deep networks by using "skip connections" or "residual blocks."

\begin{itemize}
    \item \textbf{Residual Learning:} Instead of learning the direct mapping from input to output, ResNet layers learn a residual function with reference to the layer inputs. This allows for much deeper networks that can learn complex hierarchical features.
    \item \textbf{Feature Extraction:} In the context of doodles, the initial layers of ResNet detect simple strokes and edges, while deeper layers aggregate these into shapes and eventually complete object representations.
    \item \textbf{Generalization:} We aim to test if the deep features learned from standard datasets (like QuickDraw) generalize to the high-variance, often abstract input of children's drawings.
\end{itemize}

\section{Application Demo}
To demonstrate the current capabilities of the system, we have prepared a demonstration video and screenshot.

\begin{figure}[H]
    \centering
    \includegraphics[width=0.8\textwidth]{app_demo.png}
    \caption{Interface of the Doodle Recognition App showing the Challenge Mode and Prediction Confidence scores.}
    \label{fig:app_demo}
\end{figure}

The demonstration video showcases:
\begin{enumerate}
    \item \textbf{Real-time Inference:} The ResNet model predicts the drawing as strokes are added.
    \item \textbf{Challenge Flow:} The gamified loop of prompt $\rightarrow$ draw $\rightarrow$ validate $\rightarrow$ score.
\end{enumerate}

\section{Dataset and Model Specifications}
The underlying ResNet model is trained on a comprehensive dataset of doodles.

\subsection{Dataset Overview}
\begin{itemize}
    \item \textbf{Number of Classes:} 340 categories
    \item \textbf{Total Samples:} Based on the QuickDraw dataset, encompassing millions of drawings.
    \item \textbf{Image Size:} 160x160 pixels
\end{itemize}

\subsection{Sample Data}
Below are examples of the pre-processed bitmap inputs that the model learns to recognize. These samples demonstrate the high variance and abstraction typical of doodle data.

\begin{figure}[H]
    \centering
    \begin{minipage}{0.18\textwidth}
        \centering
        \includegraphics[width=\linewidth]{sample_1.png}
        \caption*{Apple}
    \end{minipage}\hfill
    \begin{minipage}{0.18\textwidth}
        \centering
        \includegraphics[width=\linewidth]{sample_2.png}
        \caption*{Banana}
    \end{minipage}\hfill
    \begin{minipage}{0.18\textwidth}
        \centering
        \includegraphics[width=\linewidth]{sample_3.png}
        \caption*{Tree}
    \end{minipage}\hfill
    \begin{minipage}{0.18\textwidth}
        \centering
        \includegraphics[width=\linewidth]{sample_4.png}
        \caption*{Car}
    \end{minipage}\hfill
    \begin{minipage}{0.18\textwidth}
        \centering
        \includegraphics[width=\linewidth]{sample_5.png}
        \caption*{Bus}
    \end{minipage}

    \vspace{1em}

    \begin{minipage}{0.18\textwidth}
        \centering
        \includegraphics[width=\linewidth]{sample_6.png}
        \caption*{Fish}
    \end{minipage}\hfill
    \begin{minipage}{0.18\textwidth}
        \centering
        \includegraphics[width=\linewidth]{sample_7.png}
        \caption*{Flower}
    \end{minipage}\hfill
    \begin{minipage}{0.18\textwidth}
        \centering
        \includegraphics[width=\linewidth]{sample_8.png}
        \caption*{House}
    \end{minipage}\hfill
    \begin{minipage}{0.18\textwidth}
        \centering
        \includegraphics[width=\linewidth]{sample_9.png}
        \caption*{Sun}
    \end{minipage}\hfill
    \begin{minipage}{0.18\textwidth}
        \centering
        \includegraphics[width=\linewidth]{sample_10.png}
        \caption*{Star}
    \end{minipage}
    
    \caption{Representative samples from the training dataset showing the stroke-based nature of the input across 10 different categories.}
\end{figure}

\section{Conclusion}
By repositioning the Doodle Recognition app as an educational tool, we unlock a powerful use case that benefits the community while simultaneously providing valuable feedback on the generalization capabilities of our AI models.

\end{document}
